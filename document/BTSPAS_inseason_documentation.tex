\documentclass[]{article}
\usepackage{lmodern}
\usepackage{amssymb,amsmath}
\usepackage{ifxetex,ifluatex}
\usepackage{fixltx2e} % provides \textsubscript
\ifnum 0\ifxetex 1\fi\ifluatex 1\fi=0 % if pdftex
  \usepackage[T1]{fontenc}
  \usepackage[utf8]{inputenc}
\else % if luatex or xelatex
  \ifxetex
    \usepackage{mathspec}
  \else
    \usepackage{fontspec}
  \fi
  \defaultfontfeatures{Ligatures=TeX,Scale=MatchLowercase}
\fi
% use upquote if available, for straight quotes in verbatim environments
\IfFileExists{upquote.sty}{\usepackage{upquote}}{}
% use microtype if available
\IfFileExists{microtype.sty}{%
\usepackage{microtype}
\UseMicrotypeSet[protrusion]{basicmath} % disable protrusion for tt fonts
}{}
\usepackage[margin=1in]{geometry}
\usepackage{hyperref}
\hypersetup{unicode=true,
            pdftitle={Directions for Inseason Estimates for Taku River Sockeye Salmon Using BTSPAS},
            pdfauthor={Sara Miller},
            pdfborder={0 0 0},
            breaklinks=true}
\urlstyle{same}  % don't use monospace font for urls
\usepackage{graphicx,grffile}
\makeatletter
\def\maxwidth{\ifdim\Gin@nat@width>\linewidth\linewidth\else\Gin@nat@width\fi}
\def\maxheight{\ifdim\Gin@nat@height>\textheight\textheight\else\Gin@nat@height\fi}
\makeatother
% Scale images if necessary, so that they will not overflow the page
% margins by default, and it is still possible to overwrite the defaults
% using explicit options in \includegraphics[width, height, ...]{}
\setkeys{Gin}{width=\maxwidth,height=\maxheight,keepaspectratio}
\IfFileExists{parskip.sty}{%
\usepackage{parskip}
}{% else
\setlength{\parindent}{0pt}
\setlength{\parskip}{6pt plus 2pt minus 1pt}
}
\setlength{\emergencystretch}{3em}  % prevent overfull lines
\providecommand{\tightlist}{%
  \setlength{\itemsep}{0pt}\setlength{\parskip}{0pt}}
\setcounter{secnumdepth}{0}
% Redefines (sub)paragraphs to behave more like sections
\ifx\paragraph\undefined\else
\let\oldparagraph\paragraph
\renewcommand{\paragraph}[1]{\oldparagraph{#1}\mbox{}}
\fi
\ifx\subparagraph\undefined\else
\let\oldsubparagraph\subparagraph
\renewcommand{\subparagraph}[1]{\oldsubparagraph{#1}\mbox{}}
\fi

%%% Use protect on footnotes to avoid problems with footnotes in titles
\let\rmarkdownfootnote\footnote%
\def\footnote{\protect\rmarkdownfootnote}

%%% Change title format to be more compact
\usepackage{titling}

% Create subtitle command for use in maketitle
\providecommand{\subtitle}[1]{
  \posttitle{
    \begin{center}\large#1\end{center}
    }
}

\setlength{\droptitle}{-2em}

  \title{Directions for Inseason Estimates for Taku River Sockeye Salmon Using
BTSPAS}
    \pretitle{\vspace{\droptitle}\centering\huge}
  \posttitle{\par}
    \author{Sara Miller}
    \preauthor{\centering\large\emph}
  \postauthor{\par}
      \predate{\centering\large\emph}
  \postdate{\par}
    \date{April 2019}


\begin{document}
\maketitle

Reference for BTSPAS function:\\
Bonner, S. J. and Schwarz, C. J. (2019). BTSPAS: Bayesian Time
Stratified Petersen Analysis System.R package version 2019.01.07.

\section{Inseason Estimates for Taku
River}\label{inseason-estimates-for-taku-river}

After reading in the data and doing various merges, select the stat
weeks for which you want the BTSPAS to provide estimates on a FW and HW
basis.It will create a series of directories in the current workspace
that will accumulate as you run the code each week. This code will
compute the Petersen, the Full Week (FW) and Half week (HW) stratified

\subsection{Petersen using BTSPAS}\label{petersen-using-btspas}

Data files will be provided on a weekly basis from the tagging crews and
DFO commercial catch

release data - when fish are released with the following variable names
Year, TagID, ReleaseDate, ReleaseStatWeek (starts on Sunday)

recapture data from DFO - which tags are recovered in COMMERCIAL catch
only with the following names Year, TagID, RecoveryDate,
RecoveryStatWeek (starts on Sunday), RecoveryType only those records
with RecoveryType=``Commerical'' will be used

commercial catch from DFO - commercial catch EXCLUDING recoveries of
tagged fish with the following names Year, RecoveryDate,
RecoveryStatWeek CatchWithTags, RecoveryType only those records with
RecoveryType=``Commerical'' will be used. The commercial catch should
INCLUDE the count of tagged fish recovered. It is assumed that the
recovery date matches a commerical opening. For example, if a tag is
returned after the opening is closed, it is assume to have occurred
during the opening (which is usally in the first half of the week) THIS
IS IMPORTANT TO GET THE HALF WEEK ANALYSIS TO WORK. See the checks later
in the code

Download the latest version of BTSPAS from: the GitHub site at
\url{https://github.com/cschwarz-stat-sfu-ca/BTSPAS} using
devtools::install\_github(``cschwarz-stat-sfu-ca/BTSPAS'', dependencies
= TRUE, build\_vignettes = TRUE) This could take up to 20 minutes, so be
patient. (The vignettes take a long time to compile.


\end{document}
